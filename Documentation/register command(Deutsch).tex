Standardmäßig lassen sich Datensätze durch das Einlesen von YAML Dateien oder durch manuelle Eingaben in der Kommandozeile importieren.
Welche Variante geschickter ist, hängt vom Szenario ab.
Beispielsweise dürfte es in vielen Fällen schneller gehen, die Kopfdaten einer einzelnen Probe händisch einzugeben bzw. auszuwählen, anstatt zuerst eine YAML-Datei zu schreiben.
Umfangreichere Sammlungen von Datensätzen, die gegenseitig auf sich verweisen, lassen sich klarer durch YAML-Dateien darstellen.
Dabei nutzt lucentLIMS ein bestimmtes Format, welches später näher beschrieben wird.

\par

Daneben unterstützt lucentLIMS die Verwendung eingens geschriebener Import-Routinen, welche prinzipiell jedes Datenformat einlesen können.
Diese Routinen gehören nicht zum Standard und werden von Ihnen selbst gestellt.
Der Vorteil ist, dass es kein vorgegebenes Format der Eingabe-Dateien gibt und diese somit leichter lesbar sein können, als lucents generische YAML-Dateien.

\paragraph{Manuelle Eingabe / Auswahl}

Eingabe: \\
\texttt{register manual [Tabellenname]}


\paragraph{Generischer Import}

Eingabe: \\
\texttt{register generic [YAML-Datei]}

\par

Hier wird eine generisch formattierte Datei aus dem Unterverzeichnis \texttt{generic\_registration} des \textit{lucent Datenverzeichnisses} importiert.
Ein Import aus einem anderen Verzeichnis ist standardmäßig nicht vorgesehen\footnote{Falls Sie unprivilegierten Nutzern das selbstständige Ablegen von Importdateien im Datenverzeichnis ermöglichen möchten, können sie es z.B. per Samba freigeben.}, so dass die Administratoren einzulesende Dateien prüfen können.

\paragraph{Generisches Format}
Jeder Datensatz wird durch exakt ein Dokument in einer YAML-Datei repräsentiert.
Es gelten keine besonderen Regeln für unterschiedliche Tabellen.
Beim eigenständigen Erstellen dieser Dateien wird Kenntnis des Datenmodells sowie ggf. grundlegenden SQL-Statements vorausgesetzt.

\par

\subparagraph{Metadaten}
Folgende Keys beschreiben Informationen über den Datensatz selbst, enthalten jedoch nicht die eigentlichen Informationen, welche später in die Datenbank geschrieben werden sollen.

\begin{description}
\item[table name] Name jener Tabelle, in welche ein neuer Datensatz eingetragen werden soll.
Muss bereits im Vorfeld bekannt sein.
\item[before] Andere YAML-Dateien, die verarbeitet werden sollen, bevor die eigentliche Datei importiert wird.
Kann z.B. Stammdaten zum eigentlichen Datensatz enthalten.
\item[temp name] Eine vorrübergehende Bezeichnung des Datensatzes, die gespeichert wird, so lange der Import läuft.
Liefert die id des Datensatzes für andere Datensätze in der gleichen Datei, die sich auf \texttt{temp name} beziehen.
\end{description}

\subparagraph{Felder}
Der Key \texttt{fields and values} muss eine Liste von Key-Value-Paaren enthalten, welche die zu beschreibenden Felder und ihre jeweiligen Werte tragen.
Die Feldnamen müssen zwingend bekannt sein.

\par

Die Werte können entweder durch direkte Angabe oder dyamisch befüllt werden. Falls der Wert dynamisch befüllt werden soll, muss auf den Key \texttt{get by} eines der untenstehenden Wörter folgen.

\begin{description}
\item[auto generate] Kann verwendet werden, um laufende Nummern zu erzeugen.

\subitem{Option "type":} Kann auf "running", "no swap", "unique digits" oder "safe" gestellt werden (Standard: "running"). Siehe Nummerierung.

\item[sql] Ruft einen Wert durch eine SQL-Abfrage ab.
Falls mehrere Ergebnisse erzielt werden, wird nur das erste berücksichtig.

\subitem{"query":} Die SQL-Abfrage.

\item[temp name] Liefert den Wert des Primärschlüssels für einen anderen Datensatz, welcher einen temporären Namen zugeordnet bekommen hat.
\end{description}

\paragraph{Beispiel}

Hier wird eine Probe imporiert.

\lstinputlisting[language=yaml,firstnumber=1]{Samples_2022.yaml}
